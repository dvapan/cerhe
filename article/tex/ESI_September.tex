\documentclass[a4paper,12pt]{article}


\title{
  Численный метод решения
  уравнений в частных производных на примере расчета
  высокотемпературного керамического теплообменника периодического
  действия
}

\author{
Апанович Данил Владимирович
Клер Александр Матвеевич
  }

\usepackage{indentfirst} % отделять первую строку раздела абзацным отступом тоже
\usepackage[utf8x]{inputenc}
\usepackage{tempora}
\usepackage[unicode=true]{hyperref}
\usepackage{microtype}

\usepackage[english,russian]{babel}


\usepackage{mathptmx}
%\usepackage{cyrtimes}


\usepackage{amssymb,amsfonts,amsmath,amsthm} % advanced math stuff



\makeindex

\begin{document}

  \maketitle
  \section*{Введение}
  \addcontentsline{toc}{section}{Введение}

  Расчеты стационарных и нестационарных режимов работы ряда элементов
  теплоэнергетических установок (теплообменников различных типов,
  топок, камер сгорания, турбинных решеток и др.) сводятся к решению
  систем дифференциальных уравнений в частных производных
  (СДУЧП). Основными методами решения таких систем являются методы
  конечных разностей (МКР), методы контрольных объемов (МКО) и методы
  конечных элементов (МКЭ).
  
  При использовании МКР на расчетной области строится сетка и для
  каждого ее узла, на основе исходных дифференциальных уравнений,
  формируется подсистема алгебраических уравнений
  \cite{1978,Smith1985}. В этих уравнениях частные производные
  заменяются соответствующими конечными разностями. Подсистемы
  алгебраических уравнений отдельных узлов сетки объединяются в единую
  систему алгебраических уравнений, к которой добавляются краевые
  условия. Следует отметить, что при этом точность решения СДУЧП
  сильно зависит от величин шагов сетки по пространственным
  координатам и по времени. Стремление поднять точность решения
  приводит к сокращению размеров шагов и соответственно к увеличению
  числа узлов сетки и размерности системы алгебраических уравнений. Во
  многих случаях эта размерность становится столь большой, что система
  не может быть решена как единое целое без использования тех или иных
  методов декомпозиции. Снижения размерности системы можно добиться
  использованием сетки с переменными шагами, но это сильно усложняет
  алгоритм решения задачи, что особенно ощутимо для расчетного
  пространства со сложной геометрией.
  
  МКО применим к задачам, в которых дифференциальные уравнения
  отражают законы сохранения массы (полной или отдельных химических
  элементов ), энергии и импульса
  \cite{patankar1968heat,Eymard2003}). К таким задачам относится
  большинство задач тепло-массообмена. Поэтому данный метод наиболее
  широко используется в вычислительной гидрогазодинамике. В
  соответствии с МКО расчетная область разбивается на непересекающиеся
  контрольные объемы, для которых допустима неправильная
  геометрическая форма. Для каждого объема формируются балансовые
  уравнения, учитывающие обмен данного объема с соседними объемами
  массой, энергией и импульсом. Эти уравнения являются
  алгебраическими, в которых производные заменяются на конечные
  разности, определяемые по значениям соответствующих параметров в
  геометрических центрах смежных контрольных объемов. Кроме того в
  уравнения входят площади граничных поверхностей между смежными
  контрольными объемами. Причем балансы массы, энергии и импульса
  соблюдаются для контрольных объемов вне зависимости от места
  расположения разделяющих смежные объемы поверхностей. МКО позволяют
  более точно и более просто чем МКР представить сложную расчетную
  область.
  
  К недостаткам как МКР, так и МКО следует отнести невозможность
  расчета искомых переменных в точках, не являющихся узлами сетки или
  центрами контрольных объемов.
  
  МКЭ первоначально предназначались для статических расчетов
  строительных конструкций \cite{Zienkiewicz2005,johnson2012numerical}.
  Они основаны на разбиении расчетной области на достаточно большое
  число конечных элементов простой формы, как правило,
  многоугольников. На каждом элементе выделяются узлы. В первую
  очередь это вершины многоугольников, однако, возможен выбор в
  качестве узлов и других точек. Для каждого элемента для всех искомых
  из системы дифференциальных уравнений переменных, относящихся к
  данному элементу, ищутся линейные комбинации заранее заданных
  базисных функций, связывающие пространственные координаты и время с
  соответствующей искомой переменной.
  Совокупность таких комбинаций для всех элементов должна отвечать
  следующим условиям: достигается минимум суммы квадратов невязок для
  всех узлов всех конечных элементов (невязки получаются при
  подстановке в дифференциальные уравнения нужных производных
  соответствующих линейных комбинаций базисных функций); равенство
  искомых переменных в вершинах смежных элементов при определении их
  из линейных комбинаций базисных функций этих элементов; равенство
  расчетных краевых условий при определении их на основе
  соответствующих линейных комбинаций базисных функций заданным
  краевым условиям. Следует отметить, что при согласованном подборе
  числа конечных элементов, числа узлов в элементах и числа базисных
  функций можно добиться того, что невязки в узлах элементов при
  соблюдении указанных условий окажутся равными нулю. Указанные
  условия порождают систему алгебраических уравнений, решение которой
  дает линейные комбинации базисных функций, позволяющие определить
  искомые переменные в любой точке расчетной области, что является
  несомненным достоинством МКЭ. Следует отметить, что если исходная
  СДУЧП линейная, то и системы алгебраических уравнений, к которым
  сводится приближенное решение СДУЧП, будут линейными.
  
  При решении нестационарных задач с использованием МКР, МКО и МКЭ
  получающиеся системы алгебраических уравнений становятся чрезвычайно
  большими и для их решения используются методы декомпозиции,
  состоящие, как правило, в разделении решения по пространственным
  координатам и по времени. Выделяется подсистема уравнений,
  относящаяся к одному моменту времени. После ее решения находятся
  частные производные искомых величин по времени. С использованием
  этих производных определяются значения соответствующих величин в
  следующий момент времени ( на следующем временном слое ). При этом
  используются различные явные и неявные разностные схемы \cite{2012}.
  
  Во всех рассмотренных методах условием малого отклонения
  приближенного решения СДУЧП от ее точного решения является малость
  величин характерных геометрических размеров (шаги сетки,
  максимальные размеры контрольных объемов и конечных элементов
  ). Наиболее обоснованным численным критерием такого отклонения
  (качества приближенного решения ) является значение максимальной по
  модулю невязки во всех рассматриваемых (контрольных ) точках
  расчетной области, однако ни в одном из рассмотренных методов данный
  критерий не используется.
  
  С учетом указанных недостатков МКР, МКО и МКЭ предлагается более
  эффективный метод решения СДУЧП. Он основан на поиске таких значений
  коэффициентов линейных разложений базисных функций, представляющих
  зависимости искомых из СДУЧП величин от пространственных координат и
  времени, при которых минимального значения достигает максимальная по
  модулю невязка, определяемая среди всех невязок контрольных точек
  расчетной области. Переход от минимизации суммы квадратов невязок к
  минимизации максимальной по модулю невязки значительно улучшает
  качество приближенного решения и позволяет перейти от малых конечных
  элементов к достаточно крупным блокам, в пределах каждого из которых
  ищутся свои линейные разложения базисных функций. В основе метода
  лежит назначение в пределах расчетной области контрольных точек, в
  каждой из которых определяются невязки.
  
  Все контрольные точки расчетной области делятся на три
  группы. Первая группа — это внутренние контрольные точки блоков. В
  данных точках рассчитываются только невязки исходных
  дифференциальных уравнений, получающиеся после подстановки в них
  частных производных, определяемых из линейных разложений базисных
  функций. Вторая группа -это точки, лежащие на границах блоков. В
  этих точках невязки дифференциальных уравнений рассчитываются для
  каждого смежного блока с использованием его линейных
  разложений. Кроме того определяются невязки между искомыми
  величинами, а также частными производными, входящими в
  дифференциальные уравнения, рассчитываемые с помощью линейных
  разложений смежных блоков. К третьей группе относятся контрольные
  точки, лежащие на границе расчетной области. В этих точках состав
  невязок дополняется невязками, определяющими точность приближения
  полученного решения к начальным и граничным условиям. В частности,
  определяются невязки между заданными значениями величин на границах
  расчетной области и рассчитанными из линейных комбинаций базисных
  функций значениями этих величин.
  
  Следует отметить, что при минимизации максимальной по модулю невязки
  приходится сравнивать невязки, имеющие различную размерность и
  различный физический смысл. Поэтому такое сравнение целесообразно
  проводить между относительными невязками, получающимися при делении
  абсолютных невязок на их максимально допустимые значения.
  
  Если исходная СДУЧП является линейной, то предлагаемый метод,
  который можно назвать методом контрольных точек, сводится к решению
  задачи линейного программирования.


  \section{Математическая постановка задачи}
  \addcontentsline{toc}{section}{Математическая постановка задачи}

  Имеется $N$ независимых параметров $x_1,\ldots,x_N$.  В пространстве
  этих параметров задается расчетная область $Q$.  Эта область делится
  на $L$ непересекающихся подобластей $Q_1,\ldots,Q_L$. Точки, лежащие
  на границе между двумя смежными областями $Q_i$ и $Q_j$ принадлежат
  обеим подобластям. Обозначим множества граничных точек, через набор
  $\Gamma = \left\{ \Gamma_{ij} \Right\}$, где $i,j$ обозначают
  смежные подобласти $Q_i$ и $Q_j$ соответсвенно.  В каждой подобласти
  ($Q_l$) задана своя подсистема дифференциальных уравнений,
  включающая $K_l$ уравнений и $K_l$ искомых функций от независимых
  функций от независимых параметров вида $y^l_1 = y^l_1(x_1,\ldots,
  x_N),\ldots y^l_{K_l} = y^l_{K_l}(x_1,\ldots, x_N)$.

  Следует отметить, что в $k$-ое дифференциальное уравнение $l$-ой
  подсистемы дифференциальных уравнений $D^{lk}$ в общем случае входят
  не все $K_l$ искомых функций, не все $K_l \cdot N$ первых производных
  искомых функций по независимым параметрам, не все
  $K_l\frac{(N^2+N)}{2}$ вторых производных искомых функций по
  независимым параметрами и т.д. Для $k$-го уравнения $l$-ой
  подсистемы введем множество $I_{0}^{kl}$ всех номеров функций,
  входящих в $k$-ое уравнение.  Обозначим число элементов этого
  множества через $N_0^{lk}$. Очевидно, что $0 \le N_0^{lk} \le
  K_l$. Если $N_0^{lk} = 0$, то $I_0^{lk} = \emptyset$, т.е.  искомые
  функции в $k$-ое уравнение $l$-ой подсистемы не входят. $s$-й
  элемент данного множества $i_{0s}$ равен номеру $s$-ой по счету
  искомой функции, входящей в $k$-ое уравнение.

  Аналогично введем множество $I_1^{lk}$ всех первых производных,
  входящих в $k$-ое уравнение. $s$-й элемент этого множества включает
  два номера, характеризующих $s$-ую по счету первую производную
  $i^{lk}_{1s}$ --- номер искомой функции которая дифференцируется и
  $j^{lk}_{1s}$ --- номер независимого параметра по которому
  определяется соответсвующая производная. Обозначим через
  $N_{1}^{lk}$ --- число элементов в множестве $I_{1}^{lk}$. Очевидно,
  что $0 \le N_{1}^{lk} \le K_l\cdot N$.  Если $N_{1}^{lk} = 0$, то
  $I_{1}^{lk} = \emptyset$.
  

  Введем множество $I_2^{lk}$ характеризующее вторые производные,
  входящие в $k$-ое уравнение $l$-ой подсистемы. $s$-й элемент этого
  множества включает три номера, характеризующих $s$-ую по счету
  вторую производную номер искомой функции которая дифференцируется,
  номера первого и второго независимых параметров по которым
  проводится дифференцирование ($i^{lk}_{2s}$ --- номер искомой
  функции, $j^{lk}_{2s}$ --- номер первого параметра, $q^{lk}_{2s}$
  --- номер второго параметра.  Обозначим через $N_{2}^{lk}$ --- число
  элементов в множестве $I_{2}^{lk}$. Очевидно, что $0 \le N_{2}^{lk}
  \le K_l \frac{(N^2+N)}{2}$.  Если $N_{2}^{lk} = 0$, то $I_{2}^{lk} =
  \emptyset$.

  Аналогичные множества номеров и число элементов в этих множествах
  могут быть введены для производных более высоких порядков.

  С учетом данных определений система дифференциальных уравнений может
  быть записана в виде:
  

  \begin{equation}
    \label{pdegen}
    D^{lk} (
    %% Значения
    y_{i^{lk}_{01}}, y_{i^{lk}_{02}}, \ldots,y_{i^{lk}_{0N^{lk}_0}},
    %% Первые производные
    \frac{\partial y_{i^{lk}_{11}}}{\partial x_{j^{lk}_{11}}},\ldots,
    \frac{\partial y_{i^{lk}_{1N^{lk}_1}}}{\partial x_{j^{lk}_{1N^{lk}_1}}},
    %% Вторые производные
    \frac{\partial^2 y_{i^{lk}_{21}}}{\partial x_{j^{lk}_{21}}x_{q^{lk}_{21}}},\ldots,
    \frac{\partial^2 y_{i^{lk}_{2N^{lk}_1}}}{\partial x_{j^{lk}_{2N^{lk}_2}}x_{q^{lk}_{2N^{lk}_2}}},
    \ldots) = 0,
  \end{equation}

  \begin{equation*}
    k = 1,\ldots,K_l,l = 1,\ldots,L.
  \end{equation*}  

  \subsection{Граничные условия}
  \addcontentsline{toc}{subsection}{Граничные условия}

  Здесь мы опишем условия равенства значений искомых функций на
  границах смежных областей.  В каждой точке границы $\Gamma_{ij}$
  двух смежных подобластей $Q_i$ и $Q_j$ должны выполняться условия
  равенства значений части искомых функций подобласти $Q_i$
  соответсвующим им искомымх функциям подобласти $Q_j$. Введем
  множество соответсвия номеров искомых функций подобласти $Q_i$
  номерам искомых функций подобласти $Q_j$. Множество включает
  $S^{ij}$ элементов. Очевидно, что $1 \le S^{ij} \le \min{K_i, K_j}$.
  $p$-й элемент этого множества включает два номера $\omega_p^i$ и
  $\omega_p^j$, обозначающие номера искомых функций из соответсвующих
  подобластей $Q_i,Q_j$.

  Исходя из этого условия примут вид
  
  \begin{equation}
    y_{\omega_1^i}^i(x) = y_{\omega_1^j}^j(x),
    \ldots,
    y_{\omega_{S^{ij}}^i}^i(x) = y_{\omega_{S^{ij}}^j}^j(x), \forall x \in \Gamma_{ij}.
  \end{equation}

  Такие условия должны выполняться для всех наборов смежных
  подобластей \Gamma.

  Также опишем условия равенства значений искомых функций на внешних
  границах подобластей.

  Для каждой подобласти $Q_i$ вводится набор внешних границ, на
  которых задаются значения некоторых
  функций. $\Gamma_1^i,\ldots,\Gamma_{G^i}^i$ Здесь, $G^i$ Обозначает
  размерность этого набора. Для каждой границы вводится набор функций,
  значения которых на этой границе задаются в общем виде некоторой функцией от
  нескольких переменных
  

  \subsection{Замена искомых функций полиномами}
  \addcontentsline{toc}{subsection}{Замена искомых функций полиномами}


  В самом общем виде степенной полином от $N$ переменных степени не выше чем $p$ можно
  записать формулой
  
  \begin{equation}
    P_N^p(a,x_1,\ldots,x_N)=
    \displaystyle\sum_{p_1,\ldots,p_N=0}^{p}a_{p_1 \cdots p_N}\prod_{j=1}^{N}x^{p_j},
    \displaystyle\sum_{j=1}^{N}p_j \le p
  \end{equation}
  Где $a = \left\{ a_{0 \cdots 0},\ldots,a_{p_1 \cdots p_N} \right\}$.
  Мощность этого множества показывает общее число
  коэффициентов полинома от $N$ переменных, степени не выше чем $p$:
  $ \#a = C_{N+p}^N = \frac{(N+p)!}{N!p!}$.
  
  То есть в полином входят все одночлены, в которых сумма степеней
  переменных не превышает порядка полинома $p$.

  В качестве примера для 2-х переменных полином степени не выше чем 2
  будет выглядеть следующим образом:

  \begin{equation*}
    P_2^2(a,x_1,x_2) = a_{00} + a_{10}x_1 + a_{20}x_1^2+ a_{11}x_1x_2 + a_{01}x_2 + a_{02}x_2^2
  \end{equation*}

  Каждая функция, в каждой подобласти заменяется соотвествующим своим
  полиномом.

  \begin{equation}
    y_{dm}^{lk}(x_1,\ldots,x_N) = P_N^p(a^{lk},x_1,\ldots,x_N)
  \end{equation}

  Далее вводятся контрольные точки:
  $T=\left\{(x_1^1,\ldots,x_N^1),(x_1^2,\ldots,x_N^2),\ldots\right\}$
  для каждой из подобластей. Здесь верхний индекс обозначает не
  степень, а номер соответсвующего набора. Точек в подобласти для
  любого из полиномов должно быть больше чем количество его
  коэффициентов.

  В каждой точке мы вычисляем соответсвующие невязки по соответсвующим
  этой точке дифференциальным уравнениям:

  \begin{equation}
    \label{res:eq}
    {\over D}^{lkt} ( {\bar y}_{i^{lkt}_{01}}, {\bar y}_{i^{lkt}_{02}},
      \ldots, {\bar y}_{i^{lkt}_{0N^{lk}_0}},\frac{\partial {\bar
          y}_{i^{lkt}_{11}}}{\partial
        x_{j^{lk}_{11}}},\ldots,\frac{\partial {\bar
          y}_{i^{lkt}_{1N^{lk}_1}}}{\partial
        x_{j^{lk}_{1N^{lk}_1}}},\ldots) = \theta^{lkt}_1
  \end{equation}

  Здесь за $t$ обозначен индекс соответсвующей контрольной точки. За
  ${\bar y}_{i}^{lkt}$ обозначается значение полинома или его
  производной вычисленное в точке $t$.

  \begin{equation*}
    {\bar y}_i^{lkt} = y(a^{lk},x^t_1,\ldots, x^t_N)
  \end{equation*}
  
  Для тех точек из набора, что расположены на границах пар $(s,q) \in Q_{adj}$ смежных
  областей мы вычисляем невязки (в том числе и для условий реверсивности):
  
  \begin{equation}
    \label{res:adj}
    {\bar y}_i^{st}\left(x_1,x_2,\ldots,x_N\right) -
    {\bar y}_i^{qt}\left(x_1,x_2,\ldots,x_N\right) = \theta^{isqt}_2
  \end{equation}

  Для той части набора, что попадает в области, где задаются граничные
  условия мы вычисляем невязки по граничным условиям: 

  \begin{equation}
    \label{res:bnd}
    {\bar y}_i^{st}\left(x_1,x_2,\ldots,x_N\right) -
    {\bar f}_i^{t}\left(x_1,x_2,\ldots,x_N\right) = \theta^{ivt}_3
  \end{equation}
  где $v$ --- это индекс подобластей $Q_1,\ldots,Q_n \subset Q$, для
  которых имеются граничные условия.

  \subsection{Задача линейного программирования}
  \addcontentsline{toc}{subsection}{Задача линейного программирования}
   
  Таким образом мы получаем, что наборы всех невязок
  $\theta^{lkt}_1,theta^{sqt}_2,theta^{vt}_3$ зависят от коэффициентов
  полиномов, заменяющих искомые функции. Мы можем подбирать параметры
  так, чтобы минимизировать модули этих невязок, но для начала нужно учесть,
  что мы сравниваем между сосбой величины имеющие разные размерности и
  разный физический смысл. Для этого промастштабируем невязки
  соответсвующими коэффициентами $\delta^{lk}_1$ который масштабирует
  невязки, связанные с уравнениями, здесь индексы $l,k$ обозначают
  использование соответсвующего уравнения в определенной области и
  $\delta^{i}_2, \delta^{i}_3$ - который масштабирует невязки,
  связанные со значениями полиномов, заменяющий искомые функции $y_i$.

  Теперь мы можем сравнить все значения. Для этого введем параметр $z$ ---
  величину максимального отклонения. Теперь мы можем ввести условия выполнения этого ограничения.
  Для всех невязок введем:
  \begin{eqnarray}
    g^{+}_i = z - \frac{\theta_i}{\delta_i} \ge 0\\
    g^{-}_i = z + \frac{\theta_i}{\delta_i} \ge 0
  \end{eqnarray}

  Теперь мы можем сформулировать задачу оптимизации для параметров
  $a^{00}_{0 \cdots 0},\ldots,a^{00}_{p_1 \cdots p_N},\ldots,a^{lk}_{0 \cdots 0},\ldots,a^{lk}_{p_1 \cdots p_N},z$
  При учете ограничений $g^{+},g^{-}$ мы будем искать минимум $z$.

  \begin{eqnarray*}
    \min_{a^{00}_{0 \cdots 0},\ldots,a^{lk}_{p_1 \cdots p_N},z} z \\
    g^{+}_i \ge 0 \\
    g^{-}_i \ge 0
  \end{eqnarray*}
  
  \section{Расчет модели теплообменника}
  \addcontentsline{toc}{section}{Расчет модели теплообменника}

  В качестве примера работы, мы рассмотрим модель, которая описывает
  керамический теплообменник.

  \subsection{Двумерный случай}

  Мы принимаем, что каждый теплообменник представляет собой цилиндр,
  заполненный керамической засыпкой из оксида алюминия. В нашем случае
  используются два этапа - прямой и обратный. В ходе первого — мы
  нагреваем горячими газами керамический наполнитель теплообменника:

  $$(T_{gp}(x,\tau) - T_{cp}(x,\tau)) \alpha F_{spec} = - \rho^2 c^2 F_* W\frac{\partial T_{gp}(x,\tau)}{\partial x} - \rho^2 c^2 F_*\frac{\partial T_{gp}(x,\tau)}{\partial \tau}$$

  $$(T_{gp}(x,\tau) - T_{cp}(x,\tau))\alpha F_{spec}=  C_c M_{spec} \frac{\partial T_{сp}(x,\tau)}{\partial \tau}$$

  А в ходе второго - путем охлаждения керамики, нагреваем воздух

  $$(T_{gr}(x,\tau) - T_{cr}(x,\tau)) \alpha F_{spec} = - \rho^2 c^2 F_* W\frac{\partial T_{gr}(x,\tau)}{\partial x} - \rho^2 c^2 F_*\frac{\partial T_{gr}(x,\tau)}{\partial \tau}$$

  $$(T_{gr}(x,\tau) - T_{cr}(x,\tau))\alpha F_{spec}=  C_c M_{spec} \frac{\partial T_{сr}(x,\tau)}{\partial \tau}$$

  $$x \in [0, L]; \tau \in [0, T]$$

  Здесь функциями: $T_{gp}, T_{gr}, T_{cp}, T_{cr}$ обозначаются температуры:
\begin{itemize}
\item $T_{gp}$ — выхлопных газов;
\item $T_{gr}$ — воздуха;
\item $T_{cp}$ — керамики при ее нагреве;
\item $T_{cr}$ — керамики при ее остывании;
\item $\alpha$ — температуропроводность керамики;
\item $F_spec$ — удельная площадь теплообмена;
\item $\rho$ —  плотность газа (в случае прямого процесса) и воздуха в случае обратного
\item $c$ — удельная теплоемкость воздуха
\item  $C_c$  — удельная теплоемкость керамической засыпки
\item $F_*$ — площадь живого сечения для прохождения газа через засыпку
\item $W$ — скорость воздуха
\end{itemize}

Задаем следующие начальные условия

$$T_{gp}(0,\tau) = 1800$$
$$T_{gr}(L,\tau) = 778$$

Обменник работает циклически, т.е. значения температуры керамики
зависят друг от друга. Когда заканчивается процесс нагрева,
температура керамики в конце этого процесса должна быть равна
температуре керамики вначале процесса охлаждения и наоборот:

$$T_{cp}(x,T) = T_{cr}(x,0)$$
$$T_{cp}(x,0) = T_{cr}(x,T)$$

В качестве решения мы ищем такие значения
$T_{gp}, T_{gr}, T_{cp},T_{cr}$, которые удовлетворяли бы всем условиям.  Запишем все
необходимые уравнения в подходящей форме

$$(T_{gp}(x,\tau) - T_{cp}(x,\tau)) \alpha F_{spec} + \rho^2 c^2 F_* W\frac{\partial T_{gp}(x,\tau)}{\partial x} + \rho^2 c^2 F_*\frac{\partial T_{gp}(x,\tau)}{\partial \tau} = 0$$
$$(T_{gp}(x,\tau) - T_{cp}(x,\tau))\alpha F_{spec} -  C_c M_{spec} \frac{\partial T_{сp}(x,\tau)}{\partial \tau} = 0$$
$$(T_{gr}(x,\tau) - T_{cr}(x,\tau)) \alpha F_{spec} - \rho^2 c^2 F_* W\frac{\partial T_{gr}(x,\tau)}{\partial x} - \rho^2 c^2 F_*\frac{\partial T_{gr}(x,\tau)}{\partial \tau} = 0$$
$$(T_{gr}(x,\tau) - T_{cr}(x,\tau))\alpha F_{spec} -  C_c M_{spec} \frac{\partial T_{сr}(x,\tau)}{\partial \tau} = 0$$
$$x \in [0, L]; \tau \in [0, T]$$

После разбиения области $[0,L] \cross [0,T]$ на подобласти,
замены $T_{gp}, T_{gr}, T_{cp}, T_{cr}$ на соответствующие полиномы
и выбора точек в каждом регионе мы получим часть
ограничений $G_k(C_1,C_2,\ldots,C_N)$. Для второй части мы выражаем
краевые ограничения в нужной нам форме:

$$T_{gp}(0,\tau) - 1800 = 0$$
$$T_{gr}(L,\tau) - 778 = 0$$
$$T_{cp}(x,T) - T_{cr}(x,0) = 0$$
$$T_{cp}(x,0) - T_{cr}(x,T) = 0$$

Далее мы также заменяем функции на соответствующие полиномы. При этом
здесь используются только те полиномы, в подобластях которых и
установленно необходимое краевое значение.

\subsection{Трехмерный случай}

Модель, описанная выше предполагает, что тепло в наполнителе
распространяется мгновенно в каждом шарике. Далее мы проверяли влияет
ли учет или неучет этого параметра на распределение температур вцелом.
Для этого мы ввели в нашу модель учет процесса теплобмена внутри
шара. В каждой точке стержня мы добавляем один шар малого радиуса
(пока не учитывается аэродинамические показатели).  Мы рассмотрим
радиальное распространение тепла в однородном шаре
радиуса $R$. Положим, что в любой момент времени $t$ температура
на точках, находящихся на одном расстоянии $r$ от центра шара будет
одинаковой. Это означает, что температура зависит в каждом шаре только
от $r$ и от $t$. Если для каждого шара ввести сферические
координаты. В таком случае общее уравнение теплопроводности примет
вид:

$$\frac{\partial u(t,r)}{\partial \tau} - A\left(\frac{\partial^2 u(t,r)}{\partial r^2} + \frac{2}{r} \frac{\partial u(t,r)}{\partial r}\right) = 0$$

Мы считаем, что такими шарами заполнен весь стержень, кроме того на
текущем этапе считаем, что в каждом слое (конкретное значение $x$ в
функциях температуры) тепло распространяется по всем шарам равномерно,
поэтому можем считаем распределение температур для конкретного шара.

Теперь уравнения описывают функцию трех измерений, где к положению на
стержне и времени добавляется параметр радиуса. Соответственно
уравнения теперь будут отражать распространение температуры по газу и
передачу от газа к керамике для первых четырех и собственно передачу
тепла внутри наполнителя.

 $$(T_{gp}(x,\tau,r) - T_{cp}(x,\tau,r)) \alpha F_{spec} + \rho^2 c^2 F_* W\frac{\partial T_{gp}(x,\tau,r)}{\partial x} + \rho^2 c^2 F_*\frac{\partial T_{gp}(x,\tau,r)}{\partial \tau}~=~0$$
$$(T_{gp}(x,\tau,r) - T_{cp}(x,\tau,r))\alpha F_{spec} - \lambda \frac{\partial T_{сp}(x,\tau,r)}{\partial r} = 0$$
$$(T_{gr}(x,\tau,r) - T_{cr}(x,\tau,r)) \alpha F_{spec} - \rho^2 c^2 F_* W\frac{\partial T_{gr}(x,\tau,r)}{\partial x} - \rho^2 c^2 F_*\frac{\partial T_{gr}(x,\tau,r)}{\partial \tau}~=~0$$
$$(T_{gr}(x,\tau,r) - T_{cr}(x,\tau,r))\alpha F_{spec} -  \lambda \frac{\partial T_{сr}(x,\tau,r)}{\partial \tau} = 0$$
$$\frac{\partial T_{c}(x,\tau,r)}{\partial \tau} - A\left(\frac{\partial^2 T_{c}(x,\tau,r)}{\partial r^2} + \frac{2}{r} \frac{\partial T_{c}(x,\tau,r)}{\partial r}\right) = 0$$
$$x \in [0, L]; \tau \in [0, T]; r \in [r_0,r_1]$$
 
  \bibliographystyle{unsrt}
  \bibliography{refs}
\end{document}
