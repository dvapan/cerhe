\documentclass[a4paper,12pt]{article}
\usepackage{indentfirst} % отделять первую строку раздела абзацным отступом тоже
\usepackage[utf8x]{inputenc}
\usepackage{tempora}
\usepackage[unicode=true]{hyperref}
\usepackage{microtype}

\usepackage[english,russian]{babel}


\usepackage{mathptmx}
%\usepackage{cyrtimes}


\usepackage{amssymb,amsfonts,amsmath,amsthm} % advanced math stuff



\makeindex

\begin{document}

  \section*{Введение}
  \addcontentsline{toc}{section}{Введение}

  Расчеты стационарных и нестационарных режимов работы ряда элементов
  теплоэнергетических установок (теплообменников различных типов,
  топок, камер сгорания, турбинных решеток и др.) сводятся к решению
  систем дифференциальных уравнений в частных производных
  (СДУЧП). Основными методами решения таких систем являются методы
  конечных разностей (МКР), методы контрольных объемов (МКО) и методы
  конечных элементов (МКЭ).
  
  При использовании МКР на расчетной области строится сетка и для
  каждого ее узла, на основе исходных дифференциальных уравнений,
  формируется подсистема алгебраических уравнений \cite{1978,Smith1985}. В этих
  уравнениях частные производные заменяются соответствующими конечными
  разностями. Подсистемы алгебраических уравнений отдельных узлов
  сетки объединяются в единую систему алгебраических уравнений, к
  которой добавляются краевые условия. Следует отметить, что при этом
  точность решения СДУЧП сильно зависит от величин шагов сетки по
  пространственным координатам и по времени. Стремление поднять
  точность решения приводит к сокращению размеров шагов и
  соответственно к увеличению числа узлов сетки и размерности системы
  алгебраических уравнений. Во многих случаях эта размерность
  становится столь большой, что система не может быть решена как
  единое целое без использования тех или иных методов
  декомпозиции. Снижения размерности системы можно добиться
  использованием сетки с переменными шагами, но это сильно усложняет
  алгоритм решения задачи, что особенно ощутимо для расчетного
  пространства со сложной геометрией.
  
  МКО применим к задачам, в которых дифференциальные уравнения
  отражают законы сохранения массы (полной или отдельных химических
  элементов ), энергии и импульса \cite{patankar1968heat,Eymard2003}). К таким задачам относится
  большинство задач тепло-массообмена. Поэтому данный метод наиболее
  широко используется в вычислительной гидрогазодинамике. В
  соответствии с МКО расчетная область разбивается на непересекающиеся
  контрольные объемы, для которых допустима неправильная
  геометрическая форма. Для каждого объема формируются балансовые
  уравнения, учитывающие обмен данного объема с соседними объемами
  массой, энергией и импульсом. Эти уравнения являются
  алгебраическими, в которых производные заменяются на конечные
  разности, определяемые по значениям соответствующих параметров в
  геометрических центрах смежных контрольных объемов. Кроме того в
  уравнения входят площади граничных поверхностей между смежными
  контрольными объемами. Причем балансы массы, энергии и импульса
  соблюдаются для контрольных объемов вне зависимости от места
  расположения разделяющих смежные объемы поверхностей. МКО позволяют
  более точно и более просто чем МКР представить сложную расчетную
  область.
  
  К недостаткам как МКР, так и МКО следует отнести невозможность
  расчета искомых переменных в точках, не являющихся узлами сетки или
  центрами контрольных объемов.

  
  МКЭ первоначально предназначались для статических расчетов
  строительных конструкций \cite{Zienkiewicz2005,johnson2012numerical}.
  Они основаны на разбиении расчетной области на достаточно большое
  число конечных элементов простой формы, как правило,
  многоугольников. На каждом элементе выделяются узлы. В первую
  очередь это вершины многоугольников, однако, возможен выбор в
  качестве узлов и других точек. Для каждого элемента для всех искомых
  из системы дифференциальных уравнений переменных, относящихся к
  данному элементу, ищутся линейные комбинации заранее заданных
  базисных функций, связывающие пространственные координаты и время с
  соответствующей искомой переменной.
  Совокупность таких комбинаций для всех элементов должна отвечать
  следующим условиям: достигается минимум суммы квадратов невязок для
  всех узлов всех конечных элементов (невязки получаются при
  подстановке в дифференциальные уравнения нужных производных
  соответствующих линейных комбинаций базисных функций); равенство
  искомых переменных в вершинах смежных элементов при определении их
  из линейных комбинаций базисных функций этих элементов; равенство
  расчетных краевых условий при определении их на основе
  соответствующих линейных комбинаций базисных функций заданным
  краевым условиям. Следует отметить, что при согласованном подборе
  числа конечных элементов, числа узлов в элементах и числа базисных
  функций можно добиться того, что невязки в узлах элементов при
  соблюдении указанных условий окажутся равными нулю. Указанные
  условия порождают систему алгебраических уравнений, решение которой
  дает линейные комбинации базисных функций, позволяющие определить
  искомые переменные в любой точке расчетной области, что является
  несомненным достоинством МКЭ. Следует отметить, что если исходная
  СДУЧП линейная, то и системы алгебраических уравнений, к которым
  сводится приближенное решение СДУЧП, будут линейными.
  
  При решении нестационарных задач с использованием МКР, МКО и МКЭ
  получающиеся системы алгебраических уравнений становятся чрезвычайно
  большими и для их решения используются методы декомпозиции,
  состоящие, как правило, в разделении решения по пространственным
  координатам и по времени. Выделяется подсистема уравнений,
  относящаяся к одному моменту времени. После ее решения находятся
  частные производные искомых величин по времени. С использованием
  этих производных определяются значения соответствующих величин в
  следующий момент времени ( на следующем временном слое ). При этом
  используются различные явные и неявные разностные схемы \cite{2012}.
  
  Во всех рассмотренных методах условием малого отклонения
  приближенного решения СДУЧП от ее точного решения является малость
  величин характерных геометрических размеров (шаги сетки,
  максимальные размеры контрольных объемов и конечных элементов
  ). Наиболее обоснованным численным критерием такого отклонения
  (качества приближенного решения ) является значение максимальной по
  модулю невязки во всех рассматриваемых (контрольных ) точках
  расчетной области, однако ни в одном из рассмотренных методов данный
  критерий не используется.
  
  С учетом указанных недостатков МКР, МКО и МКЭ предлагается более
  эффективный метод решения СДУЧП. Он основан на поиске таких значений
  коэффициентов линейных разложений базисных функций, представляющих
  зависимости искомых из СДУЧП величин от пространственных координат и
  времени, при которых минимального значения достигает максимальная по
  модулю невязка, определяемая среди всех невязок контрольных точек
  расчетной области. Переход от минимизации суммы квадратов невязок к
  минимизации максимальной по модулю невязки значительно улучшает
  качество приближенного решения и позволяет перейти от малых конечных
  элементов к достаточно крупным блокам, в пределах каждого из которых
  ищутся свои линейные разложения базисных функций. В основе метода
  лежит назначение в пределах расчетной области контрольных точек, в
  каждой из которых определяются невязки.
  
  Все контрольные точки расчетной области делятся на три
  группы. Первая группа — это внутренние контрольные точки блоков. В
  данных точках рассчитываются только невязки исходных
  дифференциальных уравнений, получающиеся после подстановки в них
  частных производных, определяемых из линейных разложений базисных
  функций. Вторая группа -это точки, лежащие на границах блоков. В
  этих точках невязки дифференциальных уравнений рассчитываются для
  каждого смежного блока с использованием его линейных
  разложений. Кроме того определяются невязки между искомыми
  величинами, а также частными производными, входящими в
  дифференциальные уравнения, рассчитываемые с помощью линейных
  разложений смежных блоков. К третьей группе относятся контрольные
  точки, лежащие на границе расчетной области. В этих точках состав
  невязок дополняется невязками, определяющими точность приближения
  полученного решения к начальным и граничным условиям. В частности,
  определяются невязки между заданными значениями величин на границах
  расчетной области и рассчитанными из линейных комбинаций базисных
  функций значениями этих величин.
  
  Следует отметить, что при минимизации максимальной по модулю невязки
  приходится сравнивать невязки, имеющие различную размерность и
  различный физический смысл. Поэтому такое сравнение целесообразно
  проводить между относительными невязками, получающимися при делении
  абсолютных невязок на их максимально допустимые значения.
  
  Если исходная СДУЧП является линейной, то предлагаемый метод,
  который можно назвать методом контрольных точек, сводится к решению
  задачи линейного программирования.


  \section{Исходная задача}
  \addcontentsline{toc}{section}{Исходная задача}

  Имеется $N$ независимых параметров $x_1,\ldots,x_N$ и $M$ искомых
  функций $y_1,\ldots,y_M$ от независимых параметров.

  Задана расчетная область $Q$. Элементами ее являются наборы
  $(x_1,\ldots,x_N)$. Расчетная область $Q$ делится на $L$
  непересекающихся подобластей $Q_1,\ldots,Q_L$ в каждой из которых
  процессы описываются своей подсистемой дифференциальных
  уравнений \eqref{pdegen}. В общем случае, число уравнений для каждой
  подобласти различно.

  \begin{equation}
    \label{pdegen}
    D^{lk} (
    %% Значения
    y_{i^{lk}_{01}}, y_{i^{lk}_{02}}, \ldots,y_{i^{lk}_{0N^{lk}_0}},
    %% Первые производные
    \frac{\partial y_{i^{lk}_{11}}}{\partial x_{j^{lk}_{11}}},\ldots,
    \frac{\partial y_{i^{lk}_{1N^{lk}_1}}}{\partial x_{j^{lk}_{1N^{lk}_1}}},
    %% Вторые производные
    \frac{\partial^2 y_{i^{lk}_{21}}}{\partial x_{j^{lk}_{21}}x_{q^{lk}_{21}}},\ldots,
    \frac{\partial^2 y_{i^{lk}_{2N^{lk}_1}}}{\partial x_{j^{lk}_{2N^{lk}_2}}x_{q^{lk}_{2N^{lk}_2}}},
    \ldots) = 0
  \end{equation}

  В \eqref{pdegen} индекс $k = 1,\ldots K_l$ служит для перечисления
  уравнений, выполняющихся $\forall (x_1,\ldots,x_N) \in Q_l,$ в
  подситеме $l= 1,\ldots,L$, а $K_l$ --- это число уравнений для
  подобласти.

  Индексы искомых функций принимают следующие значения:

  \begin{equation*}
    i_{01}^{lk},\ldots,i_{0N_0^{lk}}^{lk} \in \left\{
    1,\ldots,M\right\}
  \end{equation*}

  Пусть тогда весь набор искомых функций без учета их производных
  следующим образом:

  \begin{equation*}
    S_0^{lk} =
    \left\{i_{01}^{lk},i_{02}^{lk},\ldots,i_{0N_0^{lk}}^{lk} \right\}
  \end{equation*}


  Для производных первого порядка индексы искомых функций и параметров
  будут выглядить:

  \begin{eqnarray*}
    i_{11}^{lk},\ldots,i_{1N_1^{lk}}^{lk} \in \left\{
    1,\ldots,M\right\}\\
    j_{11}^{lk},\ldots,j_{1N_1^{lk}}^{lk} \in \left\{
    1,\ldots,N\right\}
  \end{eqnarray*}

  Соответсвующий набор индексов для проивзодных первого порядка
  обозначим как:

  \begin{equation*}
    S_1^{lk} =
    \left\{\left(i_{11}^{lk},j_{11}^{lk}\right),\left(i_{12}^{lk},j_{12}^{lk}\right),
    \ldots,\left(i_{1N_1^{lk}}^{lk},j_{1N_1^{lk}}^{lk}\right) \right\}
  \end{equation*}

  И так далее, пока не исчерпаем все производные, входящие в \eqref{pdegen}
  

  \subsection{Граничные условия}
  \addcontentsline{toc}{subsection}{Граничные условия}

  Обозначим за $Q_{adj}$ множество пар подобластей, смежных между
  собой; $\Gamma_{sq}$ --- набор векторов, лежащих на границе между
  подобластями $s,q$.
  
  Условия равенства переменных на границах смежных областей для всех
  пар: $(s,q) \in Q_{adj}$ и для всех $\left(x_1,x_2,\ldots,x_N\right)
  \in \Gamma_{sq}$ выполняются следующие условия:

  \begin{equation}
    y_i^s\left(x_1,x_2,\ldots,x_N\right) = y_i^q\left(x_1,x_2,\ldots,x_N\right)
  \end{equation}

  Также для некоторых $x_n \in \left(x_1,x_2,\ldots,x_N\right)$ может
  выполняться условие реверсивности. Пусть параметр принимает свои
  значения на отрезке $x_n \in [a,b]$, тогда условие реверсивности запишется в виде:

  \begin{equation}
    y_i^s\left(x_1,x_2,\ldots,a,\ldots,x_N\right) = y_i^q\left(x_1,x_2,\ldots,b,\ldots,x_N\right)
  \end{equation}


  \subsection{Замена искомых функций полиномами}
  \addcontentsline{toc}{subsection}{Замена искомых функций полиномами}


  В самом общем виде степенной полином от $N$ переменных степени не выше чем $p$ можно
  записать формулой
  
  \begin{equation}
    P_N^p(a,x_1,\ldots,x_N)=
    \displaystyle\sum_{p_1,\ldots,p_N=0}^{p}a_{p_1 \cdots p_N}\prod_{j=1}^{N}x^{p_j},
    \displaystyle\sum_{j=1}^{N}p_j \le p
  \end{equation}
  Где $a = \left\{ a_{0 \cdots 0},\ldots,a_{p_1 \cdots p_N} \right\}$.
  Мощность этого множества показывает общее число
  коэффициентов полинома от $N$ переменных, степени не выше чем $p$:
  $ \#a = C_{N+p}^N = \frac{(N+p)!}{N!p!}$.
  
  То есть в полином входят все одночлены, в которых сумма степеней
  переменных не превышает порядка полинома $p$.

  В качестве примера для 2-х переменных полином степени не выше чем 2
  будет выглядеть следующим образом:

  \begin{equation*}
    P_2^2(a,x_1,x_2) = a_{00} + a_{10}x_1 + a_{20}x_1^2+ a_{11}x_1x_2 + a_{01}x_2 + a_{02}x_2^2
  \end{equation*}

  Каждая функция, в каждой подобласти заменяется соотвествующим своим
  полиномом.

  \begin{equation}
    y_{dm}^{lk}(x_1,\ldots,x_N) = P_N^p(a^{lk},x_1,\ldots,x_N)
  \end{equation}

  Далее вводятся контрольные точки:
  $(x_1^1,\ldots,x_N^1),(x_1^2,\ldots,x_N^2),\ldots$ для каждой из подобластей. Здесь верхний
  индекс обозначает не степень, а номер соответсвующего набора. Точек
  в подобласти для любого из полиномов должно быть больше чем
  количество его коэффициентов.

  В каждой точке мы вычисляем соответсвующие невязки по соответсвующим
  этой точке дифференциальным уравнениям:

  \begin{equation}
    {\over D}^{lkt} ( {\bar y}_{i^{lkt}_{01}}, {\bar y}_{i^{lkt}_{02}},
      \ldots, {\bar y}_{i^{lkt}_{0N^{lk}_0}},\frac{\partial {\bar
          y}_{i^{lkt}_{11}}}{\partial
        x_{j^{lk}_{11}}},\ldots,\frac{\partial {\bar
          y}_{i^{lkt}_{1N^{lk}_1}}}{\partial
        x_{j^{lk}_{1N^{lk}_1}}},\ldots) = \theta^{lkt}
  \end{equation}
  Здесь за $t$ обозначен индекс соответсвующей контрольной точки. За
  ${\bar y}_{i}^{lkt}$ обозначается значение полинома или его
  производной вычисленное в точке $t$.

  \begin{equation}
    {\bar y}_i^{lkt} = y(a^{lk},x^t_1,\ldots, x^t_N)
  \end{equation}
  
  \bibliographystyle{unsrt}
  \bibliography{refs}

\end{document}
